\documentclass[12pt, a4paper]{article}
\usepackage[UTF8]{ctex}
\usepackage[utf8x]{inputenc}
\usepackage{ulem}
\usepackage{hyperref}
\usepackage{graphicx}
\usepackage{listings}
\usepackage{fontspec}
\usepackage{geometry}
\geometry{left=2.5cm,right=2.5cm,top=2.5cm,bottom=2.5cm}

\begin{document}
\title{没有算法的程序优化}
\author{魏梓轩}
\maketitle

加快程序的运行速度,利用高效的算法提高产生结果的效率是一方面,编写出几乎没有缺陷的优秀代码是
另一方面。而优秀的代码,往往是算法应用和程序实现过程中最重要的一环。拿最简单的二分查找算法来
说,不同实现方法(真实坐标、偏移量)下所使用退出条件的符号也有所不同($\leq$ or $<$)。(
middle)$+1$或者$-1$操作,也尤其讲究。稍有不慎,就会陷入死循环中。简单的代码修改(如$+1$,
$-1$)有可能使得我们所写的程序成功运行,而有时候也会带来\textbf{显著的性能提升}。本文聚焦
于自己曾经吐槽,尚且可以进一步优化和尝试改过的代码 :)

\section{Man from Istria}
\begin{figure}[h!]
    \centering
    \includegraphics[width=0.3\textwidth]{asset/1021.jpg}
    \caption{\href{http://www.ituring.com.cn/book/1021}{《机器学习实战》封面}}
\end{figure}

“Man from Istria”封面插画来自于几乎人手一本的著名神书——《机器学习实战》。封面上“左擎锄,
右牵桶”的这位老大爷显然不像是一位程序员。当然,老大爷也有自己的名字,我们可以尊称他Hacquet
(1739$-$1815)。他有着令人尊敬的职业——内科医生及科学家,曾花费数年研究不同地区的植物、地
质和人种。这种精神值得大家去学习,因此这本神书以此封面所带来的寓意也十分显然了(当然,这是
我语文保持多年高分的秘诀,妄加正义的猜测=,=)。作为一本实战类书籍,书中的代码给大家提供了
一些很好的实现例子。结合着算法的数学描述,我们可以很快熟悉由理论到工程的实现过程。然而,书
中所提供的并非都是\textbf{深入工程实践}的例子,有些只是为了偷懒,而去简单的调用了一些库函
数,如\texttt{numpy.sum},\texttt{numpy.mean}等等。

如果你手头恰好有这本书,可以翻开164页(第9章,树回归)。在回归树切分函数(代码清单9-2)中,
计算数据子集切分误差的函数如下:

\begin{lstlisting}[language = Python]
  def regErr(dataSet):
    return np.var(dataSet[:, -1]) * np.shape(dataSet)[0]
\end{lstlisting}
在选取最佳切分函数\texttt{chooseBestSplit}中,计算某切分点当前误差的方法为:

\begin{lstlisting}[language = Python]
  # errType = regErr
  newS = errType(mat0) + errType(mat1)
\end{lstlisting}
在连续的实数空间中,我们选择切分点的思路一般是:1)排序;2)依次选取实数值作为切分点把数据分
为两个子集。一般来说,位于实数空间的真实数据,很难会出现几个连续的相同值。因此,切分操作对于
排序好的数据也算是“雨露均沾”(也有“独得皇上恩宠”的方法,参考XGboost)。上述代码带来的问题在
于\textbf{每次切分成两个子集\texttt{mat0}、\texttt{mat1}后,都需要\ \dotuline{从头开
始}\ 计算切分误差}。什么意思呢?我们选择一个相对好推导的“切分均值”来说明,以下是一段比较懒的
代码:

\begin{lstlisting}[language = Python]
  def regMean(dataSet):
    return np.mean(dataSet[:, -1])

  mat0, mat1 = binSplitDataset(dataSet, featIndex, splitVal)
  newM = regMean(mat0) + regMean(mat1)
\end{lstlisting}
这段代码的问题同样在于每次切分后,都需要\textbf{\dotuline{从头开始}}计算均值。而这两个
子集所发生的数据改变,仅仅是把\texttt{mat1}中的极值移动到\texttt{mat0}中。我们用
\texttt{array}表示整个数据集合,上述代码可以描述成以下数学模型:

\[
  \frac{\sum_{l=1}^{t}array_l}{t} + \frac{\sum_{r=t+1}^{N}array_r}{N-t}
\]
当我们把$array_{t+1}$的值从右子集移动到左子集,公式变成

\[
  \frac{\sum_{l=1}^{t+1}array_l}{t+1} + \frac{\sum_{r=t+2}^{N}array_r}{N-t-1}
\]
在这个过程中我们是否可以利用上一步的状态进行计算呢?那么,作进一步推导,
\begin{displaymath}
\begin{array}{ll}
    & \frac{\sum_{l=1}^{t+1}array_l}{t+1} + \frac{\sum_{r=t+2}^{N}array_r}{N-t-1} \\
  = & \frac{\sum_{l=1}^{t}array_l + array_{t+1}}{t+1} + \frac{\sum_{r=t+1}^{N}array_r - array_{t+1}}{N-t-1} \\
  = & \left(\frac{\sum_{l=1}^{t+1}array_l}{t} * \frac{t}{t+1} + \frac{array_{t+1}}{t+1}\right)
      + \left(\frac{\sum_{r=t+1}^{N}array_r}{N-t} * \frac{N-t}{N-t-1} - \frac{array_{t+1}}{N-t-1}\right)
\end{array}
\end{displaymath}
利用上一步算出的均值状态替换上式第三行中的$\sum$表达式,再对比上式第一行,可以发现经过进一步
推导式子中的\underline{加法运算}明显少了很多。这种优化在面对大型向量的时候尤为有用,而程序中
需要添加的特性仅仅是对上一步状态进行记录。学过《系统辨识》这门课的同学,可以再回顾一下“递归最小
二乘法”的精髓,我觉得就是记录状态。一个Batch进来,重新计算,太难了!

回过头来,对于“神书”所述\texttt{np.var}(方差)的计算方式,有没有递归或者记录状态的优化方式呢?
显然是有的,但是这个推导有点烦,就不放了。代码可能如下:

\begin{lstlisting}
  # 先算一个初始状态,minEventsLength为最小的区间长度
  mat0, mat1 = binSplitDataSet(dataSet, minEventsLength)
  lNum = minEventsLength
  lAvg = np.mean(mat0[:, -1])
  lSE = np.sum((mat0[:, -1] - lAvg)**2)
  rNum = m - minEventsLength
  rAvg = np.mean(mat1[:, -1])
  rSE = np.sum((mat1[:, -1] - rAvg)**2)
  # 迭代过程,灵魂都在这里
  for idx in range(minEventsLength + 1, m - minEventsLength):
    # left re-calculation
    lNum += 1
    delta = dataSet[idx, -1] - lAvg
    lAvg += delta/lNum
    newDelta = dataSet[idx, -1] - lAvg
    lSE += delta*newDelta

    # right re-calculation
    rNum -= 1
    delta = dataSet[idx, -1] - rAvg
    rAvg -= delta/rNum
    newDelta = dataSet[idx, -1] - rAvg
    rSE -= delta*newDelta

    newS = (lSE + rSE)/m
\end{lstlisting}
当然,你会觉得上面的代码阅读性差、Code Style略丑。一方面是我的原因,另一方面是高性能的代码
很难有好看的,真的 :)

\section{可能不靠谱的谱聚类}
谱聚类是靠谱的,只不过我们很有可能写出不那么靠谱的代码。注意以下公式,来自于全连接法计算邻接
矩阵$\mathbf{W}$:
\[ \omega_{ij}=\exp(-\frac{||x_i-x_j||^2_2}{2\sigma^2}) \]
我们主要关注$||x_i-x_j||^2_2$的计算。有一部分同学看到这个公式就着手去做了,写出了下面的
代码:
\begin{lstlisting}
  for row_i in feature.shape[0]:
    for row_j in feature.shape[1]:
      diff = feature[row_i, :] - feature[row_j, :] 
      W[i, j] = diff.dot(diff.T)  # 内积可能报错,但很好debug,
                                  # 至少保证ndim==2
\end{lstlisting}
我认为这段代码还可以这么写:
\begin{lstlisting}
  square_sum = (feature * feature).sum(axis=1).reshape((-1, 1))
  W = square_sum + square_sum.T - 2*feature.dot(feature.T)
\end{lstlisting}
用到了ndarray的broadcast机制。

\section{总结}
假装有总结:-(

\end{document}
